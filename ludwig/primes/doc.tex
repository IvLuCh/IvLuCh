\documentclass{article}

\usepackage{amsmath}
\usepackage{amssymb}
\usepackage[utf8]{inputenc}

\usepackage[margin=2cm]{geometry}

\begin{document}
\section{Primzahlen}
\subsection{Definition}
Eine beliebige Zahl $n \in \mathbb{N}$ ist Teil der Menge $\mathbb{P}$, wenn $n$ ausschließlich durch $1$ und $n$ teilbar ist.

\subsection{Besonderheiten}
Für eine Zahl $p \in \mathbb{P} \setminus \{2\,;\,3\}$ gilt stets:

\[p^2 = n \cdot 24 + 1;\,\,n \in \mathbb{N}\]

\subsection{Erschließung}
\paragraph{Mersennesche Primzahlen}
\[n_1 = 2^{n_2} - 1;\,\,n_2 \in \mathbb{N}_1\]
\end{document}
